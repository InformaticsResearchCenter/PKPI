\chapter{Kenaikan Pangkat}
Kenaikan pangkat merupakan suatu penghargaan yang diberikan atas kemampuan dan pengabdian asisten riset terhadap organisasi \textit{IRC}. Kenaikan pangkat dimaksudkan agar asisten riset \textit{IRC} mampu meningkatkan kemampuan dan produktivitasnya, memiliki motivasi yang lebih untuk menciptakan suatu inovasi.

\section{Syarat dan Ketentuan Naik Pangkat}
Syarat dan ketentuan kenaikan pangkat telah dimusyawarahkan dan disetujui oleh beberapa orang dosen yang terlibat dalam organisasi \textit{IRC}. Berikut syarat dan ketentuan yang harus dilaksanakan oleh asisten riset \textit{IRC}.
Syarat dan Ketentuan untuk Mendapatkan Pangkat 1 (Mitra Muda):
\begin{enumerate}
 \item Mengikuti PKM atau pelatihan sebanyak 2 kali
 \item Menerbitkan Jurnal Nasional
 \item Mengerjakan Draft HAKI
 \item Penilaian Koordinator Lab
 \item Menerbitkan Buku ber-ISBN
\end{enumerate}
Syarat dan Ketentuan untuk Mendapatkan Pangkat 2 (Mitra Tama):
\begin{enumerate}
 \item Mengikuti PKM sebanyak 3 Kali.
 \item Menerbitkan Jurnal Nasional Terakreditasi
 \item Mengikuti Perlombaan sebanyak 1 Kali
 \item Menjadi Instruktur Pelatihan 2 Kali
 \item Penilaian Koordinator Lab
 \item Menerbitkan Buku ber-ISBN
\end{enumerate}
Syarat dan Ketentuan untuk Memperoleh Pangkat 3 (Instruktur):
\begin{enumerate}
 \item Mengikuti PKM sebanyak 4 Kali
 \item Penelitian Kolaborasi Dosen 3 Kali
 \item Menerbitkan Jurnal Internasional Terakreditasi \textit{Scopus} dan lain sebagainya
 \item Penilaian Koordinator Lab
 \item Mengikuti Perlombaan sebanyak 2 Kali
 \item Menjadi Instruktur Pelatihan 4 Kali
 \item Menerbitkan Buku ber-ISBN
\end{enumerate}

\subsection{PKM (Program Kreativitas Mahasiswa)}
\par
PKM (Program Kreativitas Mahasiswa) merupakan kegiatan yang dibentuk oleh Direktorat Jendral Pembelajaran dan Kemahasiswaan Kementrian Riset sebagai suatu wadah untuk menampung kreativitas dan inovasi para mahasiswa berdasarkan Ilmu Sains dan Teknologi.
Jenis PKM (Program Kreativitas Mahasiswa) yang dapat diikuti:
\begin{enumerate}
 \item PKM-Pengabdian Kepada Masyarakat (PKM-M)
 \item PKM-Penerapan Teknologi (PKM-T)
 \item PKM-Karsa Cipta (PKM-KC)
 \item PKM-Artikel Ilmiah (PKM-AI)
 \item PKM-Gagasan Tertulis (PKM-GT)
\end{enumerate}
\par
PKM akan memberikan dampak yang sangat baik bagi seorang asisten riset. Ketika melakukan kegiatan PKM-M, asisten riset dituntut untuk bisa membagikan ilmunya kepada orang lain. Selain itu, kemampuan asisten riset dalam berbicara di depan umum akan lebih terasah, sehingga tidak ada lagi asisten riset yang tidak bisa menjadi seorang \textit{public speaker}. Begitupun dengan PKM lainnya, PKM-T yang bertujuan untuk membuat asisten riset lebih mahir di bidang teknologi. PKM-KC yang bertujuan agar seorang asisten riset dapat menciptakan suatu produk dari masalah yang ada dan produk yang memiliki daya jual tinggi. PKM-AI yang bertujuan sama halnya seperti penerbitan jurnal yaitu agar seorang asisten riset dapat menulis sebuah karya dari produk-produk yang diciptakan dan memperlihatkan hasil karyanya kepada dunia.

\subsection{Jurnal Nasional dan Internasional}
\par
Jurnal merupakan tulisan khusus yang memuat artikel suatu bidang ilmu tertentu. jurnal dibuat oleh seseorang yang berkompeten di bidangnya dan diterbitkan oleh suatu instansi.\\
\par
Tujuan asisten riset dapat membuat sebuah jurnal yaitu agar seorang asisten riset dapat menunjukkan kemampuannya kepada orang lain. Apabila seorang asisten riset menciptakan suatu produk dan ingin agar produknya tersebut diketahui oleh dunia, maka melalui jurnal asisten riset dapat memperkenalkan kemampuan dan produk-produk ciptaannya.

\subsection{Draft HAKI}
\par
HAKI (Hak Kekayaan Intelektual)adalah istilah yang dipergunakan untuk merujuk kepada seperangkat hak eksklusif yang masing-masing diberikan kepada seseorang yang telah menghasilkan karya dari olah pikirnya, yang memiliki wujud, sifat atau memenuhi kriteria tertentu berdasarkan peraturan perundang-undangan yang berlaku.\\
\par
Tujuan asisten riset dapat mengerjakan draft HAKI adalah agar asisten riset dapat menciptakan sesuatu dan menghasilkan sesuatu dari ilmu yang diperoleh. Apabila asisten riset dapat menghasilkan sesuatu dan berdaya cipta atas dirinya sendiri, maka seorang asisten riset tentulah memiliki nilai yang lebih tinggi jika dibandingkan dengan seseorang yang belum memiliki hak cipta atas dirinya sendiri.

\subsection{Penilaian Koordinator Lab}
\par
Penilaian koordinator lab merupakan penilaian mingguan yang akan dilaksanakan oleh seluruh asisten riset IRC. Penilaian koordinator lab bertujuan agar dapat mengevaluasi kinerja para asisten riset setiap minggunya dan memusyawarahkan kegiatan-kegiatan yang akan dilaksanakan untuk meningkatkan kinerja para asisten riset IRC. Penilaian koordinator lab sangat berpengaruh terhadap kenaikan pangkat, baik pangkat 1, pangkat 2, maupun pangkat 3. Dengan adanya penilaian koordinator lab diharapkan para asisten riset IRC dapat menilai kemampuan masing-masing dan termotivasi untuk lebih giat dalam meningkatkan kemampuan.

\subsection{Menerbitkan Buku Ber-ISBN}
\par
Seorang asisten riset akan diakui kemampuannya apabila telah menerbitkan buku ber-isbn. Penerbitan ini bermaksud agar asisten riset dapat berbagai ilmu yang dimiliki kepada orang lain. Ilmu tidak akan ada gunanya apabila digunakan untuk diri sendiri. Ilmu akan bermanfaat apabila seseorang dapat membagikan ilmunya tersebut kepada orang lain. Inilah tujuan dari seorang asisten riset wajib menerbitkan buku ber-isbn.

\subsection{Mengikuti Perlombaan}
\par
Mengikuti perlombaan merupakan sebuah tantangan bagi seorang asisten riset. Melalui perlombaan, asisten riset akan diuji kemampuannya oleh orang lain (juri) dan bersaing dengan oran lain atau mahasiswa-mahasiswa dari universitas lain. Melalui perlombaan, akan terlihat kualitas seorang asisten riset dalam bersaing dan kemampuannya dalam menghadapi tantangan. Perlombaan juga menguji sportivitas seorang asisten riset dan tekad yang dimiliki oleh asisten riset.

\subsection{Instruktur Pelatihan}
\par
Instruktur pelatihan adalah seseorang yang mengadakan pelatihan yang berkaitan dengan bidang keilmuannya dengan tujuan untuk membagikan ilmu yang diperoleh kepada orang lain.\\
\par 
Seorang asisten riset dituntut untuk bisa menjadi instruktur pelatihan, dengan tujuan agar asisten riset dapat membantu orang lain dalam belajar dan memahami sesuatu hingga orang yang mengikuti pelatihan dapat memahami segala sesuatu yang diajarkan oleh asisten riset dengan baik.