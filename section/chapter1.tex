\chapter{Pangkat \textit{IRC} \textit{(Informatics Research Center)}}
\par
Pangkat \textit{IRC} merupakan kedudukan yang menunjukkan tingkatan seorang asisten riset dalam susunan organisasi \textit{IRC (Informatics Research Center)}. Setiap asisten riset memiliki pangkat yang bertujuan untuk mendeskripsikan tingkatan kemampuan yang dimiliki dan pengabdian yang diberikan. 

\section{Deskripsi Pangkat}
Pangkat \textit{IRC} terdiri dari Pangkat 1 yang disebut sebagai Mitra Muda, Pangkat 2 yang disebut sebagai Mitra Tama, dan Pangkat 3 yang disebut sebagai Instruktur.

\section{Tujuan}
Berikut Tujuan diadakannya Pangkat \textit{IRC}:
\begin{enumerate}
 \item Sebagai tolok ukur atas kemampuan seorang asisten riset dalam Bidang yang ditekuni (Teknik Informatika).
 \item Mengetahui seberapa besar pengabdian seorang asisten riset kepada organisasi \textit{IRC}.
 \item Sebagai motivator dalam meningkatkan pengetahuan dan kreativitas di bidang keilmuan Teknik Informatika.
 \item Meningkatkan kerja sama antar sesama asisten riset \textit{IRC}.
 \item Sebagai pedoman dalam melaksanakan kegiatan organisasi \textit{IRC}.
\end{enumerate}
