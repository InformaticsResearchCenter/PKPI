\chapter{Form Penilaian Kenaikan Pangkat IRC}
\par
Form penilaian kenaikan pangkat IRC bertujuan untuk mengevaluasi kegiatan asisten riset IRC setiap minggunya, sehingga setiap kegiatan yang dilakukan anggota IRC akan dinilai dan dijadikan pertimbangan untuk kenaikan pangkat. penilaian ini akan didiskusikan oleh para asisten riset setiap minggunya.\\
Berikut gambaran Form Penilaian Kenaikan Pangkat IRC:
Form Penilaian Kenaikan Pangkat 1 (Mitra Tama):\\
Form penilaian pangkat 1 (Mitra Tama) berisi syarat dan ketentuan yang telah ditetapkan untuk kenaikan pangkat 1.\\
\begin{enumerate}
 \item Mengikuti PKM (Pengabdian Kepada Masyarakat) atau pelatihan. Apabila seorang asisten riset telah melakukan PKM atau pelatihan sebanyak 2 kali, maka asisten riset akan dinilai sesuai dengan kinerja yang telah ia lakukan pada saat PKM atau pelatihan berlangsung. Penilaian berupa angka 0-100 yang nantinya akan di rata-ratakan, hingga memperoleh indeks mutu (A-E) sesuai dengan nilai.
 \item Menerbitkan jurnal nasional terakreditasi. Apabila seorang asisten riset telah menerbitkan jurnal, maka jurnal yang diterbitkan oleh asisten riset akan dievaluasi sesuai dengan jurnal yang telah diterbitkan.
 \item Mengerjakan draft HAKI. Apabila seorang asisten riset telah mengerjakan draft HAKI, maka asisten riset akan dievaluasi sesuai dengan kinerja yang telah  dilakukan selama pembuatan draft HAKI.
 \item Penilaian Koordinator Lab merupakan penilaian mingguan yang diberikan oleh koordinator lab kepada asisten riset setiap minggunya.
 \item Menerbitkan buku ber-ISBN. Asisten riset akan dinilai berdasarkan buku yang telah diterbitkan dan dinilai berdasarkan kinerja asisten riset dalam mengerjakan buku yang akan diterbitkan.
\end{enumerate}
Form Penilaian Kenaikan Pangkat 2 (Mitra Muda):\\
Form penilaian pangkat 2 (Mitra Muda) berisi syarat dan ketentuan yang telah ditetapkan untuk kenaikan pangkat 2.\\
\begin{enumerate}
 \item Mengikuti PKM (Pengabdian Kepada Masyarakat) atau pelatihan. Apabila seorang asisten riset telah melakukan PKM atau pelatihan sebanyak 3 kali, maka asisten riset akan dinilai sesuai dengan kinerja yang telah ia lakukan pada saat PKM atau pelatihan berlangsung. Penilaian berupa angka 0-100 yang nantinya akan di rata-ratakan, hingga memperoleh indeks mutu (A-E) sesuai dengan nilai.
 \item Menerbitkan jurnal nasional terakreditasi. Apabila asisten riset telah menerbitkan jurnal, maka akan dievaluasi berdasarkan jurnal yang telah diterbitkan dan kinerja asisten riset selama proses pembuatan jurnal.
 \item Mengikuti Perlombaan. Penilaian dalam mengikuti perlombaan yaitu bukan berdasarkan menang atau kalahnya seorang asisten riset dalam perlombaan tersebut. Penilaian dilakukan berdasarkan usaha yang telah dilakukan oleh asisten riset dalam mengikuti perlombaan tersebut hingga selesai. Kemenangan atau Kekalahan tidak dapat mengukur seberapa besar usaha dan kinerja yang telah dilakukan oleh asisten riset.
 \item Menjadi instruktur pelatihan sebanyak 2 kali. Apabila asisten riset mengadakan pelatihan dan menjadi instruktur pelatihan sebanyak 2 kali, maka asisten riset akan dievaluasi berdasarkan kinerja asisten riset saat menjadi instruktur. Penilaian asisten riset akan dinilai dan dirata-ratakan sehingga memperoleh nilai (0-100) dan indeks nilai (A-E).
 \item Penilaian Koordinator Lab merupakan penilaian mingguan yang diberikan oleh koordinator lab kepada asisten riset setiap minggunya.
 \item Menerbitkan buku ber-ISBN. Asisten riset akan dinilai berdasarkan buku yang telah diterbitkan dan dinilai berdasarkan kinerja asisten riset dalam mengerjakan buku yang akan diterbitkan.
 \item Menerbitkan buku ber-ISBN. Asisten riset akan dinilai berdasarkan buku yang telah diterbitkan dan dinilai berdasarkan kinerja asisten riset dalam mengerjakan buku yang akan diterbitkan.
\end{enumerate}
Form Penilaian Kenaikan Pangkat 3 (Instruktur):\\
Form penilaian pangkat 3 (Instruktur) berisi syarat dan ketentuan yang telah ditetapkan untuk kenaikan pangkat 3.\\
\begin{enumerate}
 \item Mengikuti PKM (Pengabdian Kepada Masyarakat) atau pelatihan. Apabila seorang asisten riset telah melakukan PKM atau pelatihan sebanyak 4 kali, maka asisten riset akan dinilai sesuai dengan kinerja yang telah ia lakukan pada saat PKM atau pelatihan berlangsung. Penilaian berupa angka 0-100 yang nantinya akan di rata-ratakan, hingga memperoleh indeks mutu (A-E) sesuai dengan nilai.
 \item Penelitian kolaborasi dosen. Apabila seorang asisten riset mengerjakan penelitian bersama dengan dosen terkait, maka asisten riset akan dinilai kinerjanya oleh dosen tersebut dan diberikan nilai sesuai dengan yang telah dikerjakan oleh asisten riset.
 \item Menerbitkan jurnal internasional terakreditasi. Apabila asisten riset telah menerbitkan jurnal internasional, maka asisten riset akan dinilai berdasarkan kinerjanya ketika membuat jurnal internasional.
 \item Mengikuti Perlombaan. Penilaian dalam mengikuti perlombaan yaitu bukan berdasarkan menang atau kalahnya seorang asisten riset dalam perlombaan tersebut. Penilaian dilakukan berdasarkan usaha yang telah dilakukan oleh asisten riset dalam mengikuti perlombaan tersebut hingga selesai. Kemenangan atau Kekalahan tidak dapat mengukur seberapa besar usaha dan kinerja yang telah dilakukan oleh asisten riset.
 \item Menjadi instruktur pelatihan instruktur pelatihan sebanyak 3 kali. Apabila asisten riset mengadakan pelatihan dan menjadi instruktur pelatihan sebanyak 2 kali, maka asisten riset akan dievaluasi berdasarkan kinerja asisten riset saat menjadi instruktur. Penilaian asisten riset akan dinilai dan dirata-ratakan sehingga memperoleh nilai (0-100) dan indeks nilai (A-E).
 \item Menerbitkan buku ber-ISBN. Asisten riset akan dinilai berdasarkan buku yang telah diterbitkan dan dinilai berdasarkan kinerja asisten riset dalam mengerjakan buku yang akan diterbitkan.
\end{enumerate}



